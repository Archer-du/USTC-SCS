\documentclass{article}
\usepackage[zihao = -4]{ctex}
\usepackage[backend=bibtex]{biblatex}
\usepackage{listings}
\usepackage{amsmath}
\usepackage{amsfonts}
\usepackage{amssymb}
\usepackage{color}
\usepackage{tikz,pgfplots}
\usepackage{pgfplotstable}
\usepackage{longtable}
\usepackage{graphicx}
\usepackage{subfigure}
\title{ \textbf{学生选课系统需求分析}}
\author{侯甲申 PB21051110}
\date{\today}
\begin{document}
\maketitle
\tableofcontents
\section{整体需求}
本系统需要存储所有用户信息,总共有三种用户,分别为管理员、老师、学生。
每个用户都有一个用于登录系统登录的账户,
账户包括用户名和密码,同时为了防止有用户忘记密码,
还需要添加用户的身份证号和电子邮件用于找回密码或修改密码。
每个用户都包含账户、姓名、性别和唯一标识id(学号、工号);
对于学生来说,还要存储学生的课程信息(结课课程和修读课程)
和学籍信息(包括入学日期,所在院系(专业),当前学期),另外学生中
还有一类特殊的实体:助教,对于助教还要额外存储担任助教的课程信息;对于老师来说,
老师需要知道自己的任课信息(选课人数,助教信息等等);对于管理员来说,管理员需要管理所有老师
、学生、课程信息.\\
系统需要存储所有专业信息,包括唯一标识专业号、专业名称、专业简介等等,管理员需要根据不同专业
来为不同专业的学生生成可选课程表
\\
系统还要存储所有课程信息,每个课程都有唯一标识的课程号、课程名称、课程简介、
课程类型(核心通识、一般通识、专业必修、专业选修)、
授课老师信息、助教天团、理论课学时、实验课学时、
学分、所需教材、
课程评分标准(二等级、百分制、五等级制)、成绩、绩点、
开课学期、上课时间(每周几的第几节课)、
上课起始周(第几周开始上,根据上课时间即可推测出上到第几周),
此外系统还要为不同专业的学生生成各自的可选课程信息,包括
选课人数上限,已选人数.
\section{实体功能}
\subsection{管理员}
实现对所有其他实体的增删改查.
\subsection{学生}
选课、退课、查询课程所有信息、查询课程已选人数、向课程老师申请助教.
\subsection{老师}
查询任课已选学生人数和学生姓名、学号、成绩等信息,为学生登记成绩,向管理员申请任课的增删改,招收助教
\subsection{课程、专业}
为不同专业的学生生成不同的可选课程.
\end{document}